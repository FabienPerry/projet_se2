%%%%%%%%%%%%%%%%%%%%%%%%%%%%%%%%%%%%%%%%%%%%%%%%%%%%%%%%%%%%%%%%%%%%%%%%%%%%%%%
%% Si vous voulez plus petit passez en 11pt voire encore plus petit en 10pt
\documentclass[12pt]{report}

%% Je suis francophone !
\usepackage[francais]{babel}
\usepackage[utf8]{inputenc}

%% J'aime bien pouvoir contrôler mes hauts de page !
\usepackage{fancyheadings}

%% Je veux pouvoir inclure des figures...
\usepackage[pdftex]{graphicx}

%% ... des figures ``jpeg'' ou ``pdf'' ou "png"
\DeclareGraphicsExtensions{.jpg,.pdf,.png}

%% Je veux créer des Hyperdocuments
\usepackage[pdftex,colorlinks=true,linkcolor=blue,citecolor=blue,urlcolor=blue]{hyperref}

%% Je contrôle la taille de ma zone imprimée...
\usepackage{anysize}
%% ...en définissants les marges {gauche}{droite}{haute}{basse}
\marginsize{22mm}{14mm}{12mm}{25mm}

%% J'inclue une bibliographie ; j'ai donc besoin du package natbib
\usepackage{natbib}

%Pour utiliser la virgule comme séparateur décimal en mode math,
%sans que Latex introduise un espace inutile après la virgule :
\usepackage{icomma}

%%%%%%%%%%%%%%%%%%%%%%%%%%%%%%%%%%%%%%%%%%%%%%%%%%%%%%%%%%%%%%%%%%%%%%%%%%%%%%%
%% Je peux définir mes propres « commandes »...

%% pour citer une figure :
\newcommand{\figref}[1]{figure~\ref{#1}}
\newcommand{\Figref}[1]{Fig.~\ref{#1}}

%% pour citer une équation :
\newcommand{\Ref}[1]{(\ref{#1})}

%% pour mettre l'emphase sur un mot ou un groupe de mots :
\newcommand{\empha}[1]{\textit{\textbf{#1}}}

%% pour encadrer une équation :
\newcommand{\encadre}[1]{\fbox{$\displaystyle{#1}$}}

%% Symboles des unités SI :
\newcommand{\m}{\textrm{m}}
\newcommand{\s}{\textrm{s}}

%% Opérateurs de dérivation :
\newcommand{\D}{\partial}
\newcommand{\Dt}{\partial_t}
\newcommand{\Dx}{\partial_x}
\newcommand{\Dy}{\partial_y}
\newcommand{\Div}{\textrm{div}}

%% Vecteurs :
\newcommand{\Delv}{\overline{\mbox{\boldmath{$\Delta$}}}}

\newcommand{\ev}{\overline{\textbf{e}}}
\newcommand{\evr}{\ev_r}
\newcommand{\er}{\evr}
\newcommand{\evt}{\ev_\theta}
\newcommand{\ethe}{\ev_\theta}
\newcommand{\etheta}{\ev_\theta}
\newcommand{\evz}{\ev_z}
\newcommand{\ez}{\ev_z}

\newcommand{\gv}{\overline{\textbf{g}}}

\newcommand{\nv}{\overline{\textbf{n}}}
\newcommand{\Tv}{\overline{\textbf{T}}}
\newcommand{\vv}{\overline{\textbf{v}}}
\newcommand{\nabv}{\overline{\mbox{\boldmath{$\nabla$}}}}

%% Tenseur d'ordre 2 :
\newcommand{\taut}{\overline{\overline{\mbox{\boldmath{$\tau$}}}}}
\newcommand{\sigt}{\overline{\overline{\mbox{\boldmath{$\sigma$}}}}}
\newcommand{\nabt}{\overline{\overline{\mbox{\boldmath{$\nabla$}}}}}

\begin{document}
%% Voilà des hauts de page comme je les aime :
\pagestyle{fancyplain}
\renewcommand{\chaptermark}[1]{\markboth{\chaptername\ \thechapter. #1}{}}
\renewcommand{\sectionmark}[1]{\markright{\thesection. #1}}
\lhead[]{\fancyplain{}{\bfseries\leftmark}}
\rhead[]{\fancyplain{}{\bfseries\thepage}}
\cfoot{}

%% Voilà mes légendes de figures comme je les aime :
\makeatletter
\def\figurename{{\protect\sc \protect\small\bfseries Fig.}}
\def\f@ffrench{\protect\figurename\space{\protect\small\bf \thefigure}\space}
\let\fnum@figure\f@ffrench%
\let\captionORI\caption
\def\caption#1{\captionORI{\rm\small #1}}
\makeatother

%%%%%%%%%%%%%%%%%%%%%%%%%%%%%%%%%%%%%%%%%%%%%%%%%%%%%%%%%% Couverture :
\thispagestyle{empty}
{\Large
\begin{center}
\vskip1cm

%% Pour redéfinir la distance entre la boite et le texte
\fboxsep6mm
%% Pour redéfinir l'épaisseur de la boite
\fboxrule1.3pt

%% Le \vphantom{\int_\int} sert à introduire de l'espace entre les deux lignes
%% (essayez donc de le commenter)
$$\fbox{$
  \begin{array}{c}
  \textbf{Sémaphores locaux dans le Kernel}
  \vphantom{\int_\int}
  \\
  \textbf{rédigé par Jérémie Sold et Fabien dont on tairera le nom}
  \end{array}
  $}
$$
\end{center}
\vskip8cm

\begin{flushright}
Document public
\end{flushright}
}

\clearpage

%%%%%%%%%%%%%%%%%%%%%%%%%%%%%%%%%%%%%%%%%%%%%%%%%%%%%%%%%% Table des matières :
\renewcommand{\baselinestretch}{1.30}\small \normalsize

\tableofcontents

\renewcommand{\baselinestretch}{1.18}\small \normalsize


%%%%%%%%%%%%%%%%%%%%%%%%%%%%%%%%%%%%%%%%%%%%%%%%%%%%%%%%%% Introduction :
\chapter*{Introduction\markboth{Introduction}{}}
\addcontentsline{toc}{chapter}{Introduction}

Les sémaphores locaux développés dans ce package sont destinés à blablabla
\begin{center}
\href{http://emmanuelplaut.perso.univ-lorraine.fr/latex}{
\underline{http://emmanuelplaut.perso.univ-lorraine.fr/latex}
}
\end{center}
\vskip12mm

Vous êtes invités à le modifier sans vergogne~!
\vskip12mm

Bla

bla

bla

bla
\vskip3mm

blabla et rebla !
\vskip3mm

Et merci à Ludovic BUHLER pour son aide !..

\begin{flushright}
Nancy, le \today.
\vskip3mm

Emmanuel PLAUT.
\end{flushright}

%%%%%%%%%%%%%%%%%%%%%%%%%%%%%%%%%%%%%%%%%%%%%%%%%%%%%%%%%% Chap 1 :
\chapter{Présentation de l'entreprise}
Je suis enseignant à l'Université de Lorraine (UL), chercheur au LEMTA.
Mon CV résumé est disponible dans l'annexe \ref{annexe:CV}.
%%%%%%%%%%%%%%%%%%%%%%%%%%%%%%%%%%%%%%%%%%%%%%%%%%%%%%%%%%
\section{L'UL}
L'\emph{Université de Lorraine} est une grande université
comprenant dix écoles d'ingénieurs.
Je suis plus particulièrement affecté à l'une d'entre elles, l'ENSMN.


%%%%%%%%%%%%%%%%%%%%%%%%%%%%%%%%%%%%%%%%%%%%%%%%%%%%%%%%%% Chap 2 :
\chapter{Présentation du travail réalisé}
Je suis enseignant-chercheur, d'où la présentation
qui suit de mon travail en deux sections «~enseignement~» et «~recherche~».

%%%%%%%%%%%%%%%%%%%%%%%%%%%%%%%%%%%%%%%%%%%%%%%%%%%%%%%%%%
\section{Activités d'enseignements}
J'enseigne par exemple le calcul tensoriel dans le but de faire de la mécanique
des milieux continus...
Par exemple pour être capable de caractériser l'état de contraintes
d'un matériau,
c'est-à-dire de manipuler des formules du type de celle-ci
(définition du vecteur contrainte à partir du tenseur de Cauchy)~:
$$\Tv~=~\sigt\cdot\nv~.
$$
Ou comprendre d'où viennent l'équation de conservation de la masse
d'un fluide
$$\frac{\D\rho}{\D t}+\Div(\rho\vv)~=~0~,
$$
ou celle de Navier-Stokes\footnote{
J'estime tellement cette équation que je l'encadre...}
\fboxsep2mm\fboxrule.9pt
$$
  \encadre{\rho\frac{d\vv}{dt}~=~\rho\Bigg[\frac{\D\vv}{\D t}~+~
  \Big(\nabt_{\textbf{x}}\vv\Big)\cdot\vv\Bigg]~=~
  \rho\gv~-~\nabv p~+~\eta\Delv\vv}~.
$$
Voire l'ordre de grandeur de la viscosité d'un fluide
$$\eta~=~1,8~10^{-5}~\m/\s~,
$$
pour l'air dans des conditions standard.

Mes activités d'enseignements consistent aussi à aider mes élèves à se mettre à \LaTeX,
c'est pourquoi j'ai préparé ce document~!
%%%%%%%%%%%%%%%%%%%%%%%%%%%%%%%%%%%%%%%%%%%%%%%%%%%%%%%%%%
\section{Activités de recherche}
Elles ont trait à la dynamique non linéaire de systèmes étendus.
Une introduction pédagogique aux problématiques qui me motivent est présentée
dans \cite{Manneville-91}.
Récemment j'ai par exemple développé une reformulation
des contraintes de Reynolds $\taut$
engendrées par des ondes pures bidimensionnelles,
à l'ordre non linéaire le plus bas.
Ce travail, qui a fait l'objet d'une publication dans l'article
\cite{Plaut-etal-08}, permet de mieux cerner les mécanismes d'instabilités
d'écoulements cisaillés en approfondissant l'analyse de
\cite{Pedlosky-87}, ou encore de mieux comprendre la forme des
écoulements zonaux créés en thermoconvection tournante,
et qui y jouent un rôle important, cf. \cite{Morin-Dormy-06}.
En géométrie cartésienne $xy$, $x$ étant la direction de périodicité de l'onde,
cette reformulation s'écrit
\begin{equation}
  \label{tau_cart}
  \tau_{xx}~=~-2E_{cx}~,~~~\tau_{yy}~=~-2E_{cy}~,~~~
  \tau_{xy}~=~\tau_{yy}~\tan\alpha~,
\end{equation}
avec
\begin{equation}
  E_{cx}~=~\frac{1}{2}\left<v_x^2\right>_x~,~~~
  E_{cy}~=~\frac{1}{2}\left<v_y^2\right>_x
\end{equation}
les énergies cinétiques moyennes correspondant aux composantes $x$ et $y$
du champ de vitesse de l'onde,
$\alpha$ l'angle entre les séparatrices entre cellules de l'onde
et la direction $y$.
En géométrie cylindrique $r\theta$, $r$ étant le rayon cylindrique
et $\theta$ l'angle des coordonnées cylindriques,
qui est aussi la direction de périodicité de l'onde,
la reformulation est très similaire à \Ref{tau_cart},
à savoir
\begin{equation}
  \label{tau_cyl}
  \tau_{rr}~=~-2E_{cr}~,~~~\tau_{\theta\theta}~=~-2E_{c\theta}~,~~~
  \tau_{r\theta}~=~\tau_{rr}~\tan\alpha~,
\end{equation}
avec
\begin{equation}
  E_{cr}~=~\frac{1}{2}\left<v_r^2\right>_\theta~,~~~
  E_{cy}~=~\frac{1}{2}\left<v_\theta^2\right>_\theta
\end{equation}
les énergies cinétiques moyennes correspondant aux composantes $r$
et $\theta$
du champ de vitesse de l'onde,
$\alpha$ l'angle entre les séparatrices entre cellules de l'onde
et la direction $r$.
%%%%%%%%%%%%%%%%%%%%%%%%%%%%%%%%%%%%%%%%%%%%%%%%%%%%%%%%%% Conclusion :
\chapter*{Conclusion}
\addcontentsline{toc}{chapter}{Conclusion}
Je vous souhaite bonne chance dans votre apprentissage de \LaTeX...
\clearpage
%#####################################################
\appendix
\def\chaptername{Annexe}
\chapter{CV résumé}
\label{annexe:CV}
À titre d'information, je donne ici mon CV résumé.
Une version anglaise (grosso modo équivalente)
de celui-ci figure sur le site web du Lemta
\begin{center}
  \href{http://www.lemta.fr}{\underline{www.lemta.fr}}~,
\end{center}
dans la rubrique «~annuaire~».
\section{Formation initiale}
\begin{itemize}
\item[1989~:] École Polytechnique, Palaiseau, France
\vskip2mm

\item[1992~:] DEA (Master) de Physique théorique, Paris, France
\vskip2mm

\item[1996~:] Doctorat en Physique, Université d'Orsay, France
\end{itemize}
\vskip2mm

\noindent
Le sujet de ma thèse de doctorat, encadrée par Roland Ribotta,
directeur de recherche au CNRS, fut
l'\emph{Étude d'un système dynamique complexe~: thermoconvection anisotrope
en géométrie étendue}.
\section{Principaux éléments concernant la carrière}
\begin{itemize}
\item[1996-1998~:] Post-doc à l'Institut de Physique Théorique,
Université de Bayreuth, Allemagne
\vskip2mm

\item[1998-2008~:] Maître de conférences à l'École Nationale Supérieure
d'Électricité et de Mécanique (ENSEM), France
\vskip2mm

\item[2008-...~:] Professeur à l'ENSMN, France
\vskip2mm

\item[1998-...~:] Chercheur au Lemta
\end{itemize}

\section{Principale responsabilité}
Responsabilité du département Énergie \& Fluides
à Mines Nancy, en deuxième et troisième année de la formation Ingénieur Civil
de cette école.
Ce département compte environ 15 élèves en deuxième,
15 élèves en troisième année.
%%%%%%%%%%%%%%%%%%%%%%%%%%%%%%%%%%%%%%%%%%%%%%%%%%%%%%%%%% Bibliographie :

%% Base de données bibliographiques : biblio.bib
\bibliography{biblio}

\addcontentsline{toc}{chapter}{Bibliographie}

%% Feuille de style bibliographique : monjfm.bst
\bibliographystyle{monjfm}

\end{document}
